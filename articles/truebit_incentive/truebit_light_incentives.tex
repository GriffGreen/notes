\documentclass[11pt,letterpaper]{article}
\usepackage[utf8]{inputenc}
\usepackage[english]{babel}
\usepackage{latexsym,amsbsy,amssymb,amsmath,amsfonts}
\usepackage{amsthm}
\usepackage{epsfig}
\usepackage{euscript}
\usepackage{fullpage}
\usepackage{multirow,multicol,booktabs}
\usepackage[bf]{caption}
\usepackage{graphicx}
\usepackage{enumitem}
\usepackage{mathtools}
\usepackage{hyperref}
\hypersetup{
    pdfauthor={Christian Reitwießner},
    pdftitle={Truebit Light's Incentive System}
    unicode,
    breaklinks,
    colorlinks=false,
    pdfborder={0 0 0}
}

\date{}

\renewcommand{\captionfont}{\small}


%%%%%%%%%%%%%%%%%%%%%%%%%%%%%%%%%%%%%%%%%%%%%%%%%%%%%%%%%%%%%
%%%%%%%%%%%%%%%%%%%%%%%%%%%%%%%%%%%%%%%%%%%%%%%%%%%%%%%%%%%%%
%%%%%%%%%%%%%%%%%%%%%%%%%%%%%%%%%%%%%%%%%%%%%%%%%%%%%%%%%%%%%


\hfuzz=0mm
\tolerance=10000
\hbadness=1000


\setlength{\parindent}{0mm}
\setlength{\parskip}{2ex plus0.5ex minus0.5ex}


\newcommand{\Sum}{\sum\limits}
\newcommand{\Prod}{\prod\limits}


\newtheorem{dummytheorem}{Dummy-Theorem}[section]
\newtheorem{definition}[dummytheorem]{Definition}
\newtheorem{lemma}[dummytheorem]{Lemma}
\newtheorem{theorem}[dummytheorem]{Theorem}
\newtheorem{proposition}[dummytheorem]{Proposition}
\newtheorem{property}[dummytheorem]{Property}
\newtheorem{corollary}[dummytheorem]{Corollary}
\newtheorem{example}[dummytheorem]{Example}
\newtheorem{remark}[dummytheorem]{Remark}
\newtheorem{fact}[dummytheorem]{Fact}
\newtheorem{claim}[dummytheorem]{Claim}
\newtheorem{subclaim}{Subclaim}[dummytheorem]
\newtheorem{conjecture}[dummytheorem]{Conjecture}

\newcommand{\uint}{\mathbb{N}}
\newcommand{\rational}{\mathbb{Q}}

\newcommand{\oli}[1]{\overline{#1}}

%%%%%%%%%%%%%%%%%%%%%%%%%%%%%%%%%%%%%%%%%%%%%%%%%%%%%%%%%%%%%%%%%%%%%%%%%%%%%%%%
%%%%%%%%%%%%%%%%%%%%%%%%%%%%%%%%%%%%%%%%%%%%%%%%%%%%%%%%%%%%%%%%%%%%%%%%%%%%%%%%
%%%%%%%%%%%%%%%%%%%%%%%%%%%%%%%%%%%%%%%%%%%%%%%%%%%%%%%%%%%%%%%%%%%%%%%%%%%%%%%%



\newcommand{\eps}{\varepsilon}
\newcommand{\card}[1]{\##1}
\newcommand{\length}[1]{\mathrm{length}({#1})}

\newcommand{\pn}[1]{\textnormal{#1}}

\newcommand{\norm}[1]{||{#1}||}
\newcommand{\Norm}[1]{\left|\left|{#1}\right|\right|}

\DeclarePairedDelimiter\ceil{\lceil}{\rceil}
\DeclarePairedDelimiter\floor{\lfloor}{\rfloor}

%%%%%%%%%%%%%%%%%%%%%%%%%%%%%%%%%%%%%%%%%%%%%%%%%%%%%%%%%%%%%%%%%%%%%%%%%%%%%%%%
%%%%%%%%%%%%%%%%%%%%%%%%%%%%%%%%%%%%%%%%%%%%%%%%%%%%%%%%%%%%%%%%%%%%%%%%%%%%%%%%
%%%%%%%%%%%%%%%%%%%%%%%%%%%%%%%%%%%%%%%%%%%%%%%%%%%%%%%%%%%%%%%%%%%%%%%%%%%%%%%%


\begin{document}
\selectlanguage{english}


\title{Truebit Light's Incentive System}

\author{Christian Reitwießner \\ 
{\tt chris@ethereum.org}}


\maketitle


%\begin{abstract}
%abc
%\end{abstract}

Blockchains can reasonably ensure that programs are executed correctly in a decentralized setting.
They do not guarantee that a transaction is accepted, nor do they guarantee that an accepted transaction remains
accepted, but the probability of a rollback decreases as time progresses.

The main problem of blockchains is their scalability: The amount of computation that can happen per time is more or less fixed.
It especially does not increase the more participants join the network, it is rather limited by the slowest node in the
network.

TrueBit tries to solve this problem through interactive verification. It allows more or less arbitrarily complex computations to
be performed under the assumption that there is at least one honest participant. It does not require that participant
to be altruistic, though. TrueBit also includes some drawbacks, especially the drawback that transactions usually take
more time to be accepted and they can also be delayed arbitrarily by an attacker, as long as this attacker has enough
financial resources. The system favours correctness over liveness, i.e.\ as long as there is at least one honest
participant, it is impossible to include an incorrect computation/transaction, but an attacker can cause arbitrary delays for
correct computations/transactions to be included.

This article wants to specifically address the Dogecoin-Ethereum bridge, which requires blocks from the Dogecoin
blockchain to be verified inside an Ethereum smart contract. This verification is expensive when done directly,
and because of that, it is beneficial to utilize TrueBit to take the computation off-chain.

Having said that, all analyses are equally applicable to bridges between blockchains where the availability of blocks
can be reasonably assumed. This means that it can be used to e.g.\ offload processing volume from the main Ethereum
chain to another blockchain (which might even be a proof-of-authority chain) as long as all participants in that chain
rightfully assume that block data is available to all potential challengers.

We call the version of the protocol that does not make an effort to bring honest participants to the network TrueBit-light.
We assume that an honest participant is present who is altruistic to a certain degree. This includes keeping up
with the Ethereum network, paying the gas fee and having a certain amount of money to pay for an initial deposit.

We also simplify the system to a degree to only allow one parallel task. This might be extended to a constant amount
of tasks but would also make the presentation here more complex.

All timeouts are extended automatically in the case of network congestion unless it is proven that network congestion did not exist. This means that if you want
to a timeout to take effect, you have to provide a proof that several previous blocks had enough capacity to
include a potential response by the other party. Since TrueBit never makes a claim that state transitions take effect
in a finite amount of time, this is still consistent with the theory presented below.

The TrueBit contract has the following properties:

TrueBit consists of both a fact claiming component and a verification game. In both cases, we fix a mathematical function
$f \colon \Sigma^n \mapsto \Sigma^n$ which can be implemented on a given machine in $s$ elementary computation steps.
This is not a big limitation, since $f$ can be an interpreter for another machine, thus allowing it to run arbitrary programs,
which have a certain finite running time. Limiting the running time is a crucial component, although this limit can be
magnitudes higher than what is possible to compute in a single block of the underlying blockchain.

We will start by defining terms to help us deal with smart contract systems.

\begin{definition}
Let $\mathcal{A}$ be the set of all accounts and $\mathcal{T}$ the set of timestamps (e.g. the natural numbers).
We model a smart contract $C$ as a state machine that can receive inputs from $\mathcal{T} \times \mathcal{A} \times I$
(timestamp, sender and input)
and acts on this input by changing to a state from $S$ and producing an output from $O$. Here, $I$, $O$ and $S$
are specific to each smart contract type. We identify the smart contract with its state transition function
$C \colon S \times \mathcal{T} \times \mathcal{A} \times I \to S \times O$. The function is a partial function, i.e.
the machine is able to reject certain inputs. Smart contracts reject any input whose timestamp is
not larger than the previously non-rejected input. We also identify $C$ with the
iterated state transition function $C \colon (\mathcal{T} \times \mathcal{A} \times I)^* \to S \times O$,
where we assume an implied initial state $s_0$ such that $(s_0, o_0) = C(\varepsilon)$. The iterated state
transition function is then defined inductively as
$C(I_n, (t, a, i)) = C(s', t, a, i)$, where $(s', o') = C(I_n)$.

As a shorthand, we write $C\colon s \xRightarrow[t, a]{i} s' \mid o$ if $C(s, t, a, i) = (s', o)$.

A \emph{strategy} for a player $a \in \mathcal{A}$ is a function
from $S \to (\mathcal{T} \times \mathcal{I}) \cup \{ \bot \}$ (where $\bot$ means that the player
does not make a move).
For a strategy assignment $\mathcal{S} \colon \mathcal{A} \to (S \to (\mathcal{T} \times \mathcal{I}) \cup \{ \bot \})$,
a \emph{game} $g$ in a smart contract $C$ according to $\mathcal{S}$ is a sequence of moves, i.e.\ $g \in (\mathcal{T} \times \mathcal{A} \times \mathcal{I})^*$ such that
$C(g)$ is defined and either $g = \varepsilon$ (the empty game) or $g = g' \cdot (t, a, i)$ such that
$g'$ is a game in $C$ where $C(g') = (s, o)$, $\mathcal{S}(a)(s) = (t, i)$ and there is no $a' \in \mathcal{A}$ such that
$S(a') = (t',i')$, $t' < t$ and $C(g' \cdot (t', a', i'))$ is defined. The length of the game is called
the number of \emph{rounds}.

For a strategy assignment $\mathcal{S} \colon \mathcal{A} \to (S \to (\mathcal{T} \times \mathcal{I}) \cup \{ \bot \})$, we write
$C \rightsquigarrow_{\mathcal{S}} o$ if for any game $g$ in $C$ according to $\mathcal{S}$ there is
some $s$ such that $C(g) = (s,o)$. For a single strategy function
$s \colon S \to \mathcal{T} \times \mathcal{I} \cup \{ \bot \}$ for a player $a \in \mathcal{A}$
we write $C \rightsquigarrow_s o$
if $C \rightsquigarrow_{\mathcal{S}} o$ for any $\mathcal{S}$ that satisfies $\mathcal{S}(a) = s$.
\end{definition}

\begin{theorem}
For any function $f$ taking $s$ steps to compute,
there is an interactive game with two participants $a$ and $b$
implemented by a smart contract $G[a,b,\cdot,\cdot,\cdot]$ with the following properties:
\begin{enumerate}
\item it takes at most $1 + 2\log_2 s$ rounds and at most $ t_G \log_2 s$ time
(assuming no network congestion) for some intra-round timeout $t_G$
\item for any $x$ and $y$, there is always a strategy $s$ for player $a$ such that
$G[a,b,x,f(x),y] \rightsquigarrow_s f(x) $
\item for any $x$ and $y$, there is always a strategy $s$ for player $b$ such that
$G[a,b,x,y,f(x)] \rightsquigarrow_s f(x) $
\end{enumerate}
\end{theorem}
\begin{proof}
The game will keep the invariant that both players agree on the internal state of
the computation at some step $l$ but disagree about the state at step $h$.
Note that we can also work with hashes of internal states, so the data sent in each
round is not very large.

Initiall, $l=0$ and $h=s$ and the game halves the distance $h-l$ with every
second message (round).

Let $t_0$ be the timestamp at which the game is created. The initial state is

$G[a,b,x,y_a,y_b,t_0](\varepsilon) = (t_0, (0, x), (s, y_a, y_b))$

All following messages have to have a timestamp larger than the one in the state, i.e.
we have an implicit requirement that $t > t_p$.
Furthermore, we will omit the parameters of $G$ in the following. We will use
$\alpha$ for a generic accounts that can be either $a$ or $b$.

If $h - l > 1$, we ask both participants to submit what they think is the
internal state at step $\floor{\frac{h-l}{2}}$:
\begin{align}
&G[\dots]\colon
  t_p, (l, s_1), (h, s_a, s_b)
  \xRightarrow[t, \alpha]{s_2}
  t_p, (l, s_1), (h, s_a, s_b), (\alpha, s_2)
  \text{\quad for $\alpha \in \{a,b\}$}\label{eq_provide}\\
&G[\dots]\colon
  t_p, (l, s_1), (h, s_a, s_b), (a, s_2)
  \xRightarrow[t, b]{s_2'}
  \begin{cases}
    t, (\floor{\frac{h-l}{2}}, s), (h, s_a, s_b) & \text{if }s = s'\\
    t, (l, s_1), (\floor{\frac{h-l}{2}}, s_2, s_2') & \text{otherwise}\\
\end{cases}\label{eq_responda}\\
&G[\dots]\colon
  t_p, (l, s_1), (h, s_a, s_b), (b, s_2)
  \xRightarrow[t, a]{s_2'}
  \begin{cases}
    t, (\floor{\frac{h-l}{2}}, s), (h, s_a, s_b) & \text{if }s = s'\\
    t, (l, s_1), (\floor{\frac{h-l}{2}}, s_2', s_2) & \text{otherwise}\\
\end{cases}\label{eq_respondb}
\end{align}

If $h-l = 1$, the smart contract can actually perform the computation:

Let $f(s,i,p)$ be the internal state of the algorithm that computes $f$
after running a single step starting from step number $i$ and internal
state $s$ taking into account auxiliary proof data $p$ (the value
is undefined if $p$ is malformed or invalid).
\begin{align}
G[\dots]\colon t_p, (l, s_1), (l+1, s_a, s_b) \xRightarrow[t, \alpha]{p}
\bot \mid y_\alpha \text{ if } f(s_1,l,p) = s_\alpha\label{eq_final}
\end{align}

Furthermore, at a certain time $t > t_p + t_G$, a timeout can be claimed:
\begin{align}
&G[\dots]\colon
    t_p, \cdot, \cdot, (\alpha, s_2)
    \xRightarrow[t, \cdot]{} \bot \mid y_\alpha\label{eq_timeout1}\\
&G[\dots]\colon
    t_p, \cdot, \cdot
    \xRightarrow[t, \alpha]{} \bot \mid y_\alpha\label{eq_timeout2}
\end{align}

Let us now analyze the number of rounds of the game in the worst case. Note that timeouts
(i.e. mesages of type \eqref{eq_timeout1} or \eqref{eq_timeout2}) can directly end the game from
any state.
A message of type
\eqref{eq_provide} followed by either \eqref{eq_responda} or \eqref{eq_respondb} reduce
$h-l$ roughly by half. Apart from timeouts, these are the only messages possible until
$h = l+1$. At that point, only message \eqref{eq_final} is possible.

This means that if there are no timeouts, the game will require $1 + 2 \log_2 s$ messages.

Note that the timeouts for messages of type \eqref{eq_provide}, \eqref{eq_responda} and \eqref{eq_respondb}
all start at the same time. This mean that both parties have $t_G$ time to perform the
magnitude reduction of $h-l$. If this takes longer than $t_G$, anyone can step in and end the game.
This means that the game will take at most $t_G \log_2 s$ time (assuming there is an
actor who will trigger the timeout).

Finally, we argue why both players have a strategy to end the game with $f(x)$.
Due to symmetry, we only argue for player $a$.

Obviously, by responding in time, $a$ can always avert the situation that the game
ends with a timeout in a state different from $f(x)$.

If the current state of the game is $t_p, (l, s_1), (h, s_a, s_b)$, the strategy is
to send a message that contains the internal state of the algorithm computing $f$ at step
$\floor{\frac{h-l}{2}}$. In doing so, the smart contract will end up with a state
$t_p, (l, s_1), (l+1, s_a, s_b)$ where $s_1$ is the state at step $l$ and $s_a$ is the
state at step $l+1$. Since $s_b \neq s_a$ and the algorithm computing $f$ is deterministic,
$b$ cannot use a message of type \eqref{eq_final} to turn the smart contract into
state $y_b$. Instead, $a$ uses \eqref{eq_final} to make the smart contract output $y_a = f(x)$.
\end{proof}



\end{document}
    
