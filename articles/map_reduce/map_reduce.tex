\documentclass[11pt,letterpaper]{article}
\usepackage[utf8]{inputenc}
\usepackage[english]{babel}
\usepackage{latexsym,amsbsy,amssymb,amsmath,amsfonts}
\usepackage{amsthm}
\usepackage{epsfig}
\usepackage{euscript}
\usepackage{fullpage}
\usepackage{multirow,multicol,booktabs}
\usepackage[bf]{caption}
\usepackage{graphicx}
\usepackage{enumitem}
\usepackage{listings}
\usepackage{url}
\usepackage{hyperref}
\hypersetup{
 pdfauthor={Christian Reitwießner},
 pdftitle={MapReduce for plasma-Blockchains}
  unicode,
  breaklinks,
  colorlinks=false,
  pdfborder={0 0 0}
}

\date{}

\renewcommand{\captionfont}{\small}



\hfuzz=0mm
\tolerance=10000
\hbadness=1000


\setlength{\parindent}{0mm}
\setlength{\parskip}{2ex plus0.5ex minus0.5ex}


\newcommand{\Sum}{\sum\limits}
\newcommand{\Prod}{\prod\limits}


\newtheorem{dummytheorem}{Dummy-Theorem}[section]
\newtheorem{definition}[dummytheorem]{Definition}
\newtheorem{lemma}[dummytheorem]{Lemma}
\newtheorem{theorem}[dummytheorem]{Theorem}
\newtheorem{proposition}[dummytheorem]{Proposition}
\newtheorem{property}[dummytheorem]{Property}
\newtheorem{corollary}[dummytheorem]{Corollary}
\newtheorem{example}[dummytheorem]{Example}
\newtheorem{remark}[dummytheorem]{Remark}
\newtheorem{fact}[dummytheorem]{Fact}
\newtheorem{claim}[dummytheorem]{Claim}
\newtheorem{subclaim}{Subclaim}[dummytheorem]
\newtheorem{conjecture}[dummytheorem]{Conjecture}

\newcommand{\uint}{\mathbb{N}}
\newcommand{\rational}{\mathbb{Q}}

\newcommand{\oli}[1]{\overline{#1}}

%%%%%%%%%%%%%%%%%%%%%%%%%%%%%%%%%%%%%%%%%%%%%%%%%%%%%%%%%%%%%%%%%%%%%%%%%%%%%%%%
%%%%%%%%%%%%%%%%%%%%%%%%%%%%%%%%%%%%%%%%%%%%%%%%%%%%%%%%%%%%%%%%%%%%%%%%%%%%%%%%
%%%%%%%%%%%%%%%%%%%%%%%%%%%%%%%%%%%%%%%%%%%%%%%%%%%%%%%%%%%%%%%%%%%%%%%%%%%%%%%%



\newcommand{\eps}{\varepsilon}
\newcommand{\card}[1]{\##1}
\newcommand{\length}[1]{\mathrm{length}({#1})}

\newcommand{\pn}[1]{\textnormal{#1}}

\newcommand{\norm}[1]{||{#1}||}
\newcommand{\Norm}[1]{\left|\left|{#1}\right|\right|}



%%%%%%%%%%%%%%%%%%%%%%%%%%%%%%%%%%%%%%%%%%%%%%%%%%%%%%%%%%%%%%%%%%%%%%%%%%%%%%%%
%%%%%%%%%%%%%%%%%%%%%%%%%%%%%%%%%%%%%%%%%%%%%%%%%%%%%%%%%%%%%%%%%%%%%%%%%%%%%%%%
%%%%%%%%%%%%%%%%%%%%%%%%%%%%%%%%%%%%%%%%%%%%%%%%%%%%%%%%%%%%%%%%%%%%%%%%%%%%%%%%


\begin{document}
\selectlanguage{english}


\title{MapReduce for plasma-Blockchains}

\author{Christian Reitwießner\\
{\tt chris@ethereum.org}}


\maketitle

\section{Introduction}

The plasma system \cite{plasma} defines a structure of interconnected blockchains
arranged in a tree structure that promise scalable smart contracts. One of the key
ideas there is that each of the blockchains regularly store their current block hash
in their parent chain so that users can go to the parent chain and challenge
potentially invalid state transitions in the child chain.

Here, the scalabality does not only come from the fact that blockchains are relieved
from their load by creating a big number of smaller chains. Scalability is only
achieved once a user does not have to verify every single transaction that is
sent to the system. If, for example, a user only cares about a single smart
contract that resides in a single chain that is a leaf of the system,
it is sufficient for the user to verify this leaf and all nodes on the 
path to the root chain. If a transaction is commited by means of block hashes
all the way up to the root chain and there is no invalid state transition in
the chains on the way up to the root, the user can be reasonably sure that
the transaction cannot be declared invalid by other users.

This system still does not solve the scalability problem: As long as
the smart contract only ``lives'' inside a single blockchain, it can
only process a limited amount of transactions. While this might be enough
for some use-cases, a token contract can easily reach this limit.

The system would scale, if the token contract exists on all of the blockchains
and it is possible to move tokens up and down the tree. Users would have accounts
in only one or perhaps some of the chains and only watch the paths to the root
from chose chains. In such a simple model, an attacker could just select a
chain that is mostly unused, take it over, create an invalid state transition
that creates tokens out of thin air and then move these tokens up to the root.
If nobody is watching the attacked chain (or the attacker can turn off their
computers or censor their transactions), the attacker is safe as soon as he
or she is able to move the tokens far enough up.

Creating tokens out of thin air is a violation of the perceived invariants
of a token contract and thus, such invariants should be checked in each of the
chains: If we add a condition to the smart contract in each chain where the total balance
held in direct child chains of this chain is recored, then an attacker can still
create tokens in child chains, but he or she can only move tokens out of these
flawed chains up to their total balance. For users that are not interested
in these chains, the situation would not change: For them, someone took out
tokens from a pool that exist in this pool, but it is not relevant who did it.
In effect, the attacker of course steals tokens from users who that have
accounts in the child chains, but at least the impact of the attack is
confined to chains that are not properly watched. Furthermore, people who do not
want to move their tokens very often can transfer them to a chain further
towards the root. These chains likely have higher transaction fees (which
is not relevant if you just want to park your tokens), but also
provide higher security because they are watched by more people.

In the next sections we will give examples of smart contract systems
and how they can be distributed among a plasma system.

\section{Assumptions}

We start by defining a simplified view of the plasma system.

We abstract away the fraud proof mechanism and blockchains in general.

There is a number of \emph{chains}, each of which is modeled as a
computing node. Some chains are malicious, meaning they can have
invalid state transitions, ignore transactions or withhold information.
The chains are arranged in a tree and have a communication channel
to thair parent and their children, if they have some.
We assume that the root chain is not malicious.

There is a certain number of users. Some users are malicous.
We assume that each user can \emph{watch} a limited number of chains.
Non-malicious users watch some chains and
all chains on the path to the root from these chains.

We also assume that if a user watches a chain,
it cannot have any invalid state transitions for that period of time,
unless the user is also malicious or the parent chain can have
invalid state transitions.


\section{Token Contract}

TODO: I wanted to give a formal abstract treatment first,
but it is probably better to argue alongside the contract.

% We want to have a token contract that is deployed to all of the chains.
% Every account has a separate balance on each of the chains and tokens can be
% moved between the chains along the edges in the tree.
% Individually, all chains fulfill the following invariants:

% \begin{enumerate}
% \item If the balance of an account decreases, there has to be a valid signature by the owner.
% \item The sum of all balances in a chain only changes by the amounts sent to and
% received from parents and children.
% \end{enumerate}

% Of course, these invariants can be violated by malicious chains, but we should still have
% the following property:

% \begin{property}
% If a user watches all chains where he or she owns tokens and only accepts to receive
% tokens on chains of which he or she has validated the full history, then the user
% can always move the tokens to the root chain.
% \end{property}
% \begin{proof}

% \end{proof}

The following smart contract is replicated on all chains in
the system. The smart contract language used follows
the syntax and semantic of Solidity, but it has
one additional feature: Functions can be marked ``edge''.
Such functions
are executed as part of a transaction sent to an ``edge''
of the tree instead of a sigle chain. Parts of the code of these functions
are executed on the relative parent and other parts on
the relative child in sequence. The actual compiled smart contract
will use logs and Merkle proofs for synchronisation.
Inside an ``edge'' functions, the identifier
``child'' is an integer (0 or 1 for two children) identifying the
child relative to the parent.

\begin{lstlisting}
contract Token {
  mapping(address => uint) balance;
  uint[2] childBalance;

  // Regular transfer between two accounts in the same chain.
  function transfer(address recipient, uint amount) {
    require(balance[msg.sender] >= amount);
    balance[msg.sender] -= amount;
    balance[recipient] += amount;
  }

  // Moves tokens from a certain account in the parent to the
  // same account in the child chain.
  edge function transferToChild(amount) {
      parent {
        require(balance[msg.sender] >= amount);
        balance[msg.sender] -= amount;
        childBalance[child] += amount;
      }
      // After successful completion of the parent part,
      // the child part can be executed in the child.
      child {
        balance[msg.sender] += amount;
      }
  }

  edge function transferToParent(amount) {
    child {
      require(balance[msg.sender] >= amount);
      balance[msg.sender] -= amount;
    }
    parent {
      require(childBalance[child] >= amount);
      childBalance[child] -= amount;
      balance[msg.sender] += amount;
    }
  }
}
\end{lstlisting}

\begin{property}
If a non-malicious user watches all chains where he or she owns tokens and only accepts to receive
tokens on chains where he or she has either validated the full history of the chain
(including the path to the root) or
checked that sum of all balances (including child balances) matches the respective amount in the parent chain
(including the path to the root), then the user
can always move these tokens to the root chain.
\end{property}
\begin{proof}
First, let us observe that if such a user owns accepted tokens in a chain, the user
watches this chain and its parents and thus the code will executed exactly as stated.

Furthermore, note that in each situation where a balance of a user (as opposed
to a child chain) is decreased, the address ``msg.sender'' is used and thus
this only happens if the user intends to do so.

If all contracts on the way to the root executed as written,
the only reason the tokens cannot be moved to the root
is that some call to ``transferToParent'' fails. Since the same values
``amount'' is subtracted in the child chain and added in the parent chain,
the reason for such a failure has to be a failed ``require'' call.

A) The call fails in the child. The call can only fail in the child if the
user wanted to send more tokens that his or her balance is, but the user
would send exactly ``balance[user]'', so this cannot happen.

B) The call fails in the parent. In this situation, we have to have
``childBalance[child] < amount''. Remember that the user either verified
the full history of the chain or checked that the balances in the
child sum up to ``childBalance[child]'' in the parent. At least at synchronisation
points, this property is an invariant also for all other functions.
Since ``child.balance[msg.sender] == amount'' at the beginning of the
function call, we have that ``childBalance[child] >= child.balance[msg.sender]''
and thus a contradiction.

Since both situations are impossible, the user can move the tokens
to the root chain.
\end{proof}

\bibliographystyle{alpha}
\bibliography{bibliography}


\end{document}
