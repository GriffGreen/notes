\documentclass[11pt,letterpaper]{article}
\usepackage[utf8]{inputenc}
\usepackage[english]{babel}
\usepackage{latexsym,amsbsy,amssymb,amsmath,amsfonts}
\usepackage{amsthm}
\usepackage{epsfig}
\usepackage{euscript}
\usepackage{fullpage}
\usepackage{multirow,multicol,booktabs}
\usepackage[bf]{caption}
\usepackage{graphicx}
\usepackage{enumitem}
\usepackage{hyperref}
\hypersetup{
 pdfauthor={Vitalik Buterin, Christian Reitwießner},
 pdftitle={Updating zkSNARK Public Parameters}
  unicode,
  breaklinks,
  colorlinks=false,
  pdfborder={0 0 0}
}

\date{}

\renewcommand{\captionfont}{\small}


%%%%%%%%%%%%%%%%%%%%%%%%%%%%%%%%%%%%%%%%%%%%%%%%%%%%%%%%%%%%%
%%%%%%%%%%%%%%%%%%%%%%%%%%%%%%%%%%%%%%%%%%%%%%%%%%%%%%%%%%%%%
%%%%%%%%%%%%%%%%%%%%%%%%%%%%%%%%%%%%%%%%%%%%%%%%%%%%%%%%%%%%%


\hfuzz=0mm
\tolerance=10000
\hbadness=1000


\setlength{\parindent}{0mm}
\setlength{\parskip}{2ex plus0.5ex minus0.5ex}


\newcommand{\Sum}{\sum\limits}
\newcommand{\Prod}{\prod\limits}


\newtheorem{dummytheorem}{Dummy-Theorem}[section]
\newtheorem{definition}[dummytheorem]{Definition}
\newtheorem{lemma}[dummytheorem]{Lemma}
\newtheorem{theorem}[dummytheorem]{Theorem}
\newtheorem{proposition}[dummytheorem]{Proposition}
\newtheorem{property}[dummytheorem]{Property}
\newtheorem{corollary}[dummytheorem]{Corollary}
\newtheorem{example}[dummytheorem]{Example}
\newtheorem{remark}[dummytheorem]{Remark}
\newtheorem{fact}[dummytheorem]{Fact}
\newtheorem{claim}[dummytheorem]{Claim}
\newtheorem{subclaim}{Subclaim}[dummytheorem]
\newtheorem{conjecture}[dummytheorem]{Conjecture}

\newcommand{\uint}{\mathbb{N}}
\newcommand{\rational}{\mathbb{Q}}

\newcommand{\oli}[1]{\overline{#1}}

%%%%%%%%%%%%%%%%%%%%%%%%%%%%%%%%%%%%%%%%%%%%%%%%%%%%%%%%%%%%%%%%%%%%%%%%%%%%%%%%
%%%%%%%%%%%%%%%%%%%%%%%%%%%%%%%%%%%%%%%%%%%%%%%%%%%%%%%%%%%%%%%%%%%%%%%%%%%%%%%%
%%%%%%%%%%%%%%%%%%%%%%%%%%%%%%%%%%%%%%%%%%%%%%%%%%%%%%%%%%%%%%%%%%%%%%%%%%%%%%%%



\newcommand{\eps}{\varepsilon}
\newcommand{\card}[1]{\##1}
\newcommand{\length}[1]{\mathrm{length}({#1})}

\newcommand{\pn}[1]{\textnormal{#1}}

\newcommand{\norm}[1]{||{#1}||}
\newcommand{\Norm}[1]{\left|\left|{#1}\right|\right|}



%%%%%%%%%%%%%%%%%%%%%%%%%%%%%%%%%%%%%%%%%%%%%%%%%%%%%%%%%%%%%%%%%%%%%%%%%%%%%%%%
%%%%%%%%%%%%%%%%%%%%%%%%%%%%%%%%%%%%%%%%%%%%%%%%%%%%%%%%%%%%%%%%%%%%%%%%%%%%%%%%
%%%%%%%%%%%%%%%%%%%%%%%%%%%%%%%%%%%%%%%%%%%%%%%%%%%%%%%%%%%%%%%%%%%%%%%%%%%%%%%%


\begin{document}
\selectlanguage{english}


\title{Updating zkSNARK Public Parameters}

\author{Vitalik Buterin\\
{\tt vitalik@ethereum.org}
\and
Christian Reitwießner\\
{\tt chris@ethereum.org}}


\maketitle


\begin{abstract}
\noindent 
\end{abstract}



\section{GGPR}

Remember that in the QSP we are given polynomials $v_{0}$, \dots,$v_{m}$, $w_{0}$, \dots,$w_{m}$, a target polynomial $t$ (of degree at most $d$) and a binary input string $u$. The prover finds $a_{1}$, \dots,$a_{m}$, $b_{1}$, \dots,$b_{m}$ (that are somewhat restricted depending on $u$) and a polynomial $h$ such that
\[
t h = (v_{0} + a_{1}v_{1} +  \dots  + a_{m}v_{m}) (w_{0} + b_{1}w_{1} +  \dots  + b_{m}w_{m}).
\]


In the previous section, we already explained how the common reference string (CRS) is set up. We choose secret numbers $s$ and $\alpha$  and publish
\begin{quote}
$E(s^{0})$, $E(s^{1})$,  \dots, $E(s^{d})$ \quad and \quad $E(\alpha s^{0})$, $E(\alpha s^{1})$,  \dots, $E(\alpha s^{d})$
\end{quote}

Because we do not have a single polynomial, but sets of polynomials that are fixed for the problem, we also publish the evaluated polynomials right away:
\begin{itemize}
\item $E(t(s))$, $E(\alpha  t(s))$,
\item $E(v_{0}(s))$,  \dots, $E(v_{m}(s))$, $E(\alpha  v_{0}(s))$,  \dots, $E(\alpha  v_{m}(s))$,
\item $E(w_{0}(s))$,  \dots, $E(w_{m}(s))$, $E(\alpha  w_{0}(s))$,  \dots, $E(\alpha  w_{m}(s))$,
\end{itemize}


and we need further secret numbers $\beta _{v}$, $\beta _{w}$, $\gamma$  (they will be used to verify that those polynomials were evaluated and not some arbitrary polynomials) and publish
\begin{itemize}
\item $E(\gamma )$, $E(\beta _{v} \gamma )$, $E(\beta _{w} \gamma )$,
\item $E(\beta _{v} v_{1}(s))$,  \dots, $E(\beta _{v} v_{m}(s))$
\item $E(\beta _{w} w_{1}(s))$,  \dots, $E(\beta _{w} w_{m}(s))$
\item $E(\beta _{v} t(s))$, $E(\beta _{w} t(s))$
\end{itemize}


This is the full common reference string. In practical implementations, some elements of the CRS are not needed, but that would complicate the presentation.


\end{document}
